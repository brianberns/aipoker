\title{Kuhn Poker (AKQ Game) }
\author{Max Chiswick}
%\date{\today}

\documentclass[12pt]{article}
\usepackage{amsmath}



\begin{document}
\maketitle

%\begin{abstract}
%This is the paper's abstract \ldots
%\end{abstract}

\section{The Game}

\subsection{Rules}
\begin{itemize}
\item 2 players, each is dealt a card in {A, K, Q}
\item Each antes 1 at the start of the hand
\item Each has 1 remaining for betting
\item There is one betting round
\item The highest card is the best (i.e., A $>$ K $>$ Q)
\end{itemize}

\subsection{Betting Sequences}
\begin{itemize}
\item Action starts with P1, who can Bet 1 or Check
\item If P1 bets, P2 can either Call or Fold
\item If P1 checks, P2 can either Bet or Check
\item If P2 bets after P1 checks, P1 can then Call or Fold
\end{itemize}

\subsection{Payoffs}
\begin{itemize}
\item If a player folds to a bet, the other player wins the pot of 2 (profit of 1)
\item If both players check, the highest card player wins the pot of 2 (profit of 1)
\item If there is a bet and call, the highest card player wins the pot of 4 (profit of 2)
\end{itemize}

\section{Variables}

\subsection{P1 opening action}
P1 should never bet the K card here because if he bets the K, P2 with Q will always fold (since the lowest card can never win) and P2 with A will always call (since the best card will always win). 
By checking the K always, P1 can try to induce a bluff from P2 when P2 has the Q. 

Therefore we assign P1's strategy:
\begin{itemize}
\item Bet Q: $x$
\item Bet K: $0$
\item Bet A: $y$
\end{itemize}

\subsection{P2 after P1 bet}
After P1 bets, P2 should always call with the A and always fold the Q as explained above. 

Therefore we assign P2's strategy after P1 bet:
\begin{itemize}
\item Call Q: $0$
\item Call K: $a$
\item Call A: $1$
\end{itemize}

\subsection{P2 after P1 check}
After P1 checks, P2 should never bet with the K for the same reason as P1 should never initially bet with the K. 
P2 should always bet with the A because it is the best hand and there is no bluff to induce by checking. 

Therefore we assign P2's strategy after P1 check:
\begin{itemize}
\item Bet Q: $b$
\item Bet K: $0$
\item Bet A: $1$
\end{itemize}

\subsection{P1 after P1 check and P2 bet}
This case is similar to P2's actions after P1's bet. P1 can never call here with the worst hand (Q) and must always call with the best hand (A). 

Therefore we assign P1's strategy after P1 check and P2 bet:
\begin{itemize}
\item Call Q: $0$
\item Call K: $z$
\item Call A: $1$
\end{itemize}


So we now have 5 different variables $x, y, z, a, b$ to represent the unknown probabilities. 

\section{Solving}

\subsection{Solving for $a$}
We start by solving for $a$, how often P2 should call with a K facing a bet from P1. 
P2 should call $a$ to make P1 indifferent to bluffing (i.e., betting or checking) with card Q. 

\par If P1 checks with card Q, he will always fold afterwards (because it is the worst card and can never win). 

\begin{align}
   u(\textrm{P1 check with Q}) &= 0      
\end{align}


\par If P1 bets with card Q, 

\begin{align}
u(\textrm{P1 bet with Q}) &= (\textrm{P2 has A and always calls/wins}) + \\ (\textrm{P2 has K and folds}) + (\textrm{P2 has K and calls/wins}) \\
&= (1/2)(-1) + (1/2)[(a)(-1) + (1-a)(2)] \\
&= -1/2 - 1/2 * a + (1-a) \\
&= 1/2 - 3/2 * a
\end{align}

Setting the probabilities of betting with Q and checking with Q equal, we have:
\begin{align}
   0 &= \frac{1}{2} - \frac{3}{2} * a \\        
   \frac{3}{2} * a  &= \frac{1}{2} \\
    a &= \frac{1}{3}
\end{align}


\end{document}